
%Lo que hacen es generar grafos y una flecha para poder hacer los dibujos en caso de que quieran usarlo para explicar los algoritmos en el trabajo escrito. Un ejemplo del uso es, el siguiente:

\documentclass{article}

\usepackage{tikz}

\newcommand{\DrawGraph}[4]{

	\begin{scope}[#4]
    \foreach \pos/\nodo in {{(0,0)/1}, {(2,1)/2}, {(4,1)/3}, {(0,2)/4}, {(3,0)/5}, {(2,-1)/6}, {(4,-1)/7}}
    	\node[vertex] (#3\nodo) at \pos {\nodo};
    
    \foreach \start/\end in {1/4, 1/2, 1/6,2/5,2/3,2/6,5/7,3/7,4/2,6/7}
    	\path[edge] (#3\start) -- (#3\end);
        
    \foreach \nodo in {#1}
    	\node[selected vertex] at (#3\nodo) {\nodo};
    
    \begin{pgfonlayer}{background}
    	\foreach \start/\end in {#2}
    		\path[selected edge] (#3\start) -- (#3\end);
    \end{pgfonlayer}
    \end{scope}
    
}

\newcommand{\DrawArrow}[1]{
	\begin{scope}[scale=0.5,#1]
		\filldraw[arrow] (0,0) -- (2,0) -- +(270:0.5) -- (3,0.5) -- (2,1.5) -- +(270:0.5) -- (0,1) -- cycle;
	\end{scope}
}

\begin{document}
\tikzstyle{vertex}=[circle,fill=blue!25,minimum size=10pt,inner sep=0pt,font = \tiny]
\tikzstyle{selected vertex} = [vertex, fill=red!24, font = \tiny]
\tikzstyle{edge} = [draw,thick,-]
\tikzstyle{weight} = [font=\small]
\tikzstyle{selected edge} = [draw,line width=3pt,-,red!50]
\tikzstyle{ignored edge} = [draw,line width=3pt,-,black!20]
\tikzstyle{arrow} = [fill=red!80, draw = black!20]
\pgfdeclarelayer{background}
\pgfsetlayers{background,main}

\begin{tikzpicture}[node distance = 4cm, scale = 0.5]

\DrawGraph{1,2}{1/2}{a}{}
\DrawArrow{xshift=10cm}
\DrawGraph{1,2,3}{1/2,1/4}{b}{xshift=7.7cm}
\DrawArrow{xshift = 25.5cm}
\DrawGraph{1,2,3}{1/2,1/4,2/3}{c}{xshift=15.4cm}
    
\end{tikzpicture}
\end{document}